\documentclass{sigchi-ext}
% Please be sure that you have the dependencies (i.e., additional
% LaTeX packages) to compile this example.
\usepackage[T1]{fontenc}
\usepackage{textcomp}
\usepackage[scaled=.92]{helvet} % for proper fonts
\usepackage{graphicx} % for EPS use the graphics package instead
\usepackage{balance}  % for useful for balancing the last columns
\usepackage{booktabs} % for pretty table rules
\usepackage{ccicons}  % for Creative Commons citation icons
\usepackage{ragged2e} % for tighter hyphenation

% Some optional stuff you might like/need.
% \usepackage{marginnote} 
% \usepackage[shortlabels]{enumitem}
% \usepackage{paralist}
% \usepackage[utf8]{inputenc} % for a UTF8 editor only

%% EXAMPLE BEGIN -- HOW TO OVERRIDE THE DEFAULT COPYRIGHT STRIP --
% \copyrightinfo{Permission to make digital or hard copies of all or
% part of this work for personal or classroom use is granted without
% fee provided that copies are not made or distributed for profit or
% commercial advantage and that copies bear this notice and the full
% citation on the first page. Copyrights for components of this work
% owned by others than ACM must be honored. Abstracting with credit is
% permitted. To copy otherwise, or republish, to post on servers or to
% redistribute to lists, requires prior specific permission and/or a
% fee. Request permissions from permissions@acm.org.\\
% {\emph{CHI'14}}, April 26--May 1, 2014, Toronto, Canada. \\
% Copyright \copyright~2014 ACM ISBN/14/04...\$15.00. \\
% DOI string from ACM form confirmation}
%% EXAMPLE END

% Paper metadata (use plain text, for PDF inclusion and later
% re-using, if desired).  Use \emtpyauthor when submitting for review
% so you remain anonymous.
\def\plaintitle{Ritchie the DeskBuddy}
\def\plainauthor{Dhananjai Hariharan, Tiago Justino}
\def\emptyauthor{}
\def\plainkeywords{Authors' choice; of terms; separated; by
  semicolons; include commas, within terms only; required.}
\def\plaingeneralterms{Documentation, Standardization}

\title{Ritchie the DeskBuddy}

\numberofauthors{2}
% Notice how author names are alternately typesetted to appear ordered
% in 2-column format; i.e., the first 4 autors on the first column and
% the other 4 auhors on the second column. Actually, it's up to you to
% strictly adhere to this author notation.
\author{%
  \alignauthor{%
    \textbf{Dhananjai Hariharan}\\
    \affaddr{Rochester Institute of Technology} \\
    \affaddr{Rochester, NY 14623, USA} \\
    \affaddr{dh1723@rit.edu} }\alignauthor{%
    \textbf{Tiago Justino}\\
    \affaddr{Rochester institute of Technology} \\
    \affaddr{Rochester, NY, 14623, USA}\\
    \email{tvj6825@rit.edu} } }

% Make sure hyperref comes last of your loaded packages, to give it a
% fighting chance of not being over-written, since its job is to
% redefine many LaTeX commands.
\definecolor{linkColor}{RGB}{6,125,233}
\hypersetup{%
  pdftitle={\plaintitle},
%  pdfauthor={\plainauthor},
  pdfauthor={\emptyauthor},
  pdfkeywords={\plainkeywords},
  bookmarksnumbered,
  pdfstartview={FitH},
  colorlinks,
  citecolor=black,
  filecolor=black,
  linkcolor=black,
  urlcolor=linkColor,
  breaklinks=true,
}

% \reversemarginpar%

\begin{document}

\maketitle

% Uncomment to disable hyphenation (not recommended)
% https://twitter.com/anjirokhan/status/546046683331973120
\RaggedRight{} 

% Do not change the page size or page settings.
\begin{abstract}
  TODO
\end{abstract}

\keywords{\plainkeywords}

\category{H.5.m}{Information interfaces and presentation (e.g.,
  HCI)}{Miscellaneous}\category{See}{\url{http://acm.org/about/class/1998/}}{for
  full list of ACM classifiers. This section is required.}

\section{Introduction}

There are a wide variety of resources available to people to keep track of
events and activities, remind them about these, and to make them more
productive. Two approaches are commonly used: 1. using physical reminders such
as post-it notes and diaries. The drawback here is that the user is physically
constrained. For example if the user has post-it notes at home, these notes
aren't accessible when she is in the library; and 2. using software reminders,
such as Google Keep and Google Calendar. Although they are available to use for
free, these are easy to ignore as they only exist virtually (only provide
simple notifications) and are not reflected sufficiently enough in the physical
world.

This project attempts a new approach to this application. What if it were
possible for a physical object in a user's space that could draw attention and
spark user reaction in a more effective manner? What if there could be an
object that can play the role of a friend that reminds a person about the
occurrence of an event? Wouldn't it be great if someone were to tell you to
stop what you were doing and keep up with your new year resolution of running
5K everyday?

Ritchie the DeskBuddy is just the device for the role. Ritchie is a 3D printed
robotic tiger that sits on a user's desk and reminds them of events or
activities that need to be done. As an internet-connected device, Ritchie can
be of great utility when provided with pertinent data sources. To be clear,
Ritchie is not an AI robot. It cannot tell you what you need to do, or tell you
about the weather. Nor can it specifically remind you to buy some eggs on your
way back home. What Ritchie can do is wave its arms and move around to get your
attention about something important. The rest is upto the user to check on the
topic of concern. The events can range anything from small everyday events to
more important things that demand immediate action.

%\marginpar{%
%  \vspace{-45pt} \fbox{%
%    \begin{minipage}{0.925\marginparwidth}
%      \textbf{Good Utilization of the Side Bar} \\
%      \vspace{1pc} \textbf{Preparation:} Do not change the margin
%      dimensions and do not flow the margin text to the
%      next page. \\
%      \vspace{1pc} \textbf{Materials:} The margin box must not intrude
%      or overflow into the header or the footer, or the gutter space
%      between the margin paragraph and the main left column. The text
%      in this text box should remain the same size as the body
%      text. Use the \texttt{{\textbackslash}vspace{}} command to set
%      the margin
%      note's position. \\
%      \vspace{1pc} \textbf{Images \& Figures:} Practically anything
%      can be put in the margin if it fits. Use the
%      \texttt{{\textbackslash}marginparwidth} constant to set the
%      width of the figure, table, minipage, or whatever you are trying
%      to fit in this skinny space.
%    \end{minipage}}\label{sec:sidebar} }

%\begin{figure}
%  \includegraphics[width=0.9\columnwidth]{figures/sigchi-logo}
%  \caption{Insert a caption below each figure.}~\label{fig:sample}
%\end{figure}

% \begin{figure}
%   \includegraphics[width=.9\columnwidth]{figures/ea-figure2}
%   \caption{If your figure has a light background, you can set its
%     outline to light gray, like this, to make a box around
%     it.}\label{fig:bats}
% \end{figure}

%\begin{marginfigure}[-35pc]
%  \begin{minipage}{\marginparwidth}
%    \centering
%    %\includegraphics[width=0.9\marginparwidth]{figures/cats}
%    \caption{In this image, the cats are tessellated within a square
%      frame. Images should also have captions and be within the
%      boundaries of the sidebar on page~\pageref{sec:sidebar}. Photo:
%      \cczero~jofish on Flickr.}~\label{fig:marginfig}
%  \end{minipage}
%\end{marginfigure}

%\begin{figure*}
%  \centering
%  \includegraphics[width=1.4\columnwidth]{figures/map}
%  \caption{In this image, the map maximizes use of space. You can make
%    figures as wide as you need, up to a maximum of the full width of
%    both columns. Note that \LaTeX\ tends to render large figures on a
%    dedicated page. Image: \ccbynd~ayman on Flickr.}~\label{fig:cats}
%\end{figure*}

%\marginpar{\vspace{-23pc}So long as you don't type outside the right
%  margin or bleed into the gutter, it's okay to put annotations over
%  here on the left, too; this annotation is near Hawaii. You'll have
%  to manually align the margin paragraphs to your \LaTeX\ floats using
%  the \texttt{{\textbackslash}vspace{}} command.}

%\begin{margintable}[1pc]
%  \begin{minipage}{\marginparwidth}
%    \centering
%    \begin{tabular}{r r l}
%      & {\small \textbf{First}}
%      & {\small \textbf{Location}} \\
%      \toprule
%      Child & 22.5 & Melbourne \\
%      Adult & 22.0 & Bogot\'a \\
%      \midrule
%      Gene & 22.0 & Palo Alto \\
%      John & 34.5 & Minneapolis \\
%      \bottomrule
%    \end{tabular}
%    \caption{A simple narrow table in the left margin
%      space.}~\label{tab:table2}
%  \end{minipage}
%\end{margintable}

\nocite{peek2009hangsters}
\nocite{jafarinaimi2005breakaway}
\nocite{rogers2010ambient}
\nocite{fortmann2013make}
\nocite{ishii1997tangible}

\balance{}

\bibliographystyle{SIGCHI-Reference-Format}
\bibliography{references}

\end{document}

%%% Local Variables:
%%% mode: latex
%%% TeX-master: t
%%% End:
